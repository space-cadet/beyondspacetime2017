\documentclass[11pt]{article}

\usepackage{geometry}
\usepackage[T1]{fontenc}
\usepackage[latin9]{inputenc}
\usepackage{amsmath}
\usepackage{amssymb}

\usepackage{hyperref}

%\usepackage{enumitem}
\usepackage[ampersand]{easylist}
\usepackage{parskip}


\title{Referee Reports}

\begin{document}

\maketitle

\tableofcontents

\section{Beyond Spacetime Essay Contest, Dec 15, 2017}

\section*{Recommendation}

We want to thank you for submitting your paper to the 2017 Space and Time Beyond Quantum Gravity Essay Contest. We regret to inform you that you are not one of the winners this year. Please find the comments from the referees below.

\section*{Referee 1}

\textbf{This is a lovely small article, suitable for being awarded with a prize.} It makes a very good basic observation concerning the way Strings and Loops could be reconciled. It gives a remarkably synthetic account of a core results for both sides, and compares the two.  It is an article that would raise attention and interest (and polemics) if awarded.  This would be good for the prize, for physics and for the dialogue between communities.  On the technical side, my opinion is that the article centers the key point of the issue at play. 

\section*{Referee 2}

The idea is that string theory can be made ``background independent'' by computing the worldsheet area in the N-G action using the area operator of LQG ? the area is the sum of areas associated with the piercings of the worldsheet by network edges. 

\begin{easylist}
& As I understand the situation though, the LQG areas correspond to spacelike hypersurfaces, while the string is not spacelike. In short, the two areas are incommensurable.

& Even if I am wrong abut this, nothing is really said about the physical consequences - for instance, 

&& What of the string spectrum?

&& How is a path integral to be computed? So the paper lacks details.

& Finally, while the paper does make conceptual points, some are not clear and some are fairly commonplace - and the central point is the scientific proposal, which would be better placed in a specialist journal.
\end{easylist}

\section*{Referee 3}

Interesting essay with a proposal for unification of LQG and string theory. The proposal, in a nutshell, is to do a pre-geometric quantization of the Nambu-Goto string, in the spirit of LQG.

\textbf{The proposal is interesting and deserves further work.} However, as a research paper, I find the proposal to be both insufficiently motivated/evidence provided for and insufficiently developed.

\begin{easylist}
& First, we are not told what the pre-geometric state (26) is supposed to look like or even how it is obtained (not even conceptually, let alone technically).

&& Are we supposed to do a pre-geometric quantization of the target space manifold? If so, what would that look like?

&& If the answer is no and there is no target space (classical or quantum), can we still call this a string theory? (there would be no maps from world-sheet to target space).

& Other questions: (25) assumes the group SU(2).

&& But why is this the relevant group for a string theory on a manifold of Lorentzian signature?

&& The author gives no motivation at all for why this is the right formula, i.e. how this formula is derived. So, both conceptually and technically, the conjecture is insufficiently motivated.

& Some other comments: On p.2, 't Hooft cannot really be taken as representative of the string theory viewpoint since he has always been critical about strings.

& The supposed similarities on p.2 between LQG and string theory are somewhat superficial.

&& It is for all we know simply not true that ``fundamental degrees of freedom are the same'' between the two theories.

&& Same for discrete geometry: in string theory, one gets discrete geometry only in very specific situations (matrix theory).

&& Even in T-duality, it is not the geometry that is quantized, but the strings on it. And so on. I think the attempt at ecumenism on p.2 puts the accent in the wrong places.

&& On the other hand, no mention at all is made of recent progress (by Smolin, Verlinde, and others) in connecting the two theories: though the Ryu-Takayanagi relation, holography, etc.

\end{easylist}

\section{SciPost, Jan 19 \& 21, 2018}

\section*{Referee 1}

\paragraph{Strengths} None

\paragraph{Weaknesses} No results

\paragraph{Report} This is a weak paper. It starts by pointing out some superficial connections between string theory and loop quantum gravity (LQG), such as the fact that they both contain one dimensional objects. None of these connections stand up to closer scrutiny.
The author then reviews some basic facts about LQG, such as the area operator and its eigenvalues before turning to the Nambu-Goto action, which is describes a bosonic string, and a very brief review the connection between the Einstein equations and fixed point of RG is presented.
The fun starts in Section 3.2. The polyakov action is replaced with an area which is quantised using the eigenvalues from LGQ. And then nothing is done with this. It's not clear to me that this even defines a theory (a functional of fields is replaced by a function of integers). If sense can be made of this, there's no reason to believe that this has anything to do with the Nambu-Goto action, or reproduces any of its physics.
In short this is a very speculative paper that is almost certainly wrong. I cannot recommend publication in a serious journal like SciPost.

\paragraph{Requested changes} Cannot be saved.

validity: poor; significance: poor; originality: poor; clarity: poor; formatting: perfect; grammar: reasonable

\section*{Referee 2}

\paragraph{Strengths} None

\paragraph{Weaknesses} 1. The paper has no significant new physics content. 2. Some of the statements in the paper are incorrect.

\paragraph{Report}  This paper reviews some basic aspects of loop quantum gravity (the spin network version thereof) and string theory. Since people have
defines something called the area operator for spin networks, and the Nambu-Goto action is given by the area, the author suggests that we should perhaps replace the string action by the area operator and in this way combine the two frameworks. Besides saying these words the paper has no non-trivial content. No evidence is provided that the idea makes sense, there are many obvious technical issues that are not addressed at all, and the main motivation seems to be the misguided notion that string theory is not background independent.

\paragraph{Requested changes} None. Paper should be rejected.

validity: poor; significance: poor; originality: poor; clarity: low; formatting: acceptable; grammar: reasonable

\section{Modern Physics Letters A, July 19, 2019}

\section*{Referee 1}

The author tries to establish a connection between string theory and loop quantum gravity. Section 1 contains some heuristic arguments which are vague at best and partially incorrect, at the very least not at the level of a serious discussion in a professional scientific journal, e.g. the part about matter coupling or mapping the state spaces.

Sections 2-3 review basic textbook knowledge. 

Section 4 contains the main argument, which is the claim that the area operator in LQG is somehow related to the string world sheet. First, this is technically implausible because the LQG area operator refers to spatial codimension 2 hypersurfaces, while the string worldsheet is always dimension 2 and contains a timelike direction (this is also mentioned in section 5). I cannot find any plausible argument why those two concepts should be related, except that their dimension agrees for LQG formulated in 4 spacetime dimensions. The introduction of $ T_{loop} $ in section 4.1 also appears without reason, as one can always write $ T_{string} $ as the product of two numbers. The author doesn't explain the physical meaning of $ T_{loop} $. 

In conclusion, I cannot find convincing arguments in the draft that LQG and string theory should be related along the lines of the proposal of the author. Hence, I cannot recommend publication. 

\section*{Referee 2}

I cannot recommend this paper for publication because it contains flawed and incomplete arguments. To begin with, I very strongly disagree with section 3.2. First of all, it is not true that string theory is background dependent. The author bases his argument on the Polyakov action, which indeed assumes a Minkowski (or Euclidean) flat background. But string field theory, both in its open and closed formulations, does not rely on any background at all, nor on any geometry. It is as "pre-geometric" as one might wish. Second, there is absolutely nothing wrong with a theory of quantum gravity postulating a smooth, continuous spacetime, and stating the contrary reflects only prejudice rather than a scientific argument. Consider all perturbative QFT approaches to quantum gravity, where gravitons propagate on a (curved or flat) background: successful examples are effective QG by Donoghue, asymptotic safety with resummed propagator by Ward, Stelle theory, which is
renormalizable but nonunitary, or nonlocal quantum gravity. And others.

String field theory is, moreover, the correct framework where to consider interacting worldsheets, rather than the vague picture conjectures in section 4. The arguments of this section are too unsubstantiated to make progress from them, and there also seems to be inconsistencies or, at the very least, deeply obscure points. For instance, strings carrying SU(2) representations are open ones, but these are not associated with gravity. Also, (27) could resemble the kinetic term of string field theory (with the operator A replaced by the BRST operator Q), were it not that the left-hand side refers to one non-interacting string. It is hard to make sense of this.

There are also some minor issues, which however do not affect my recommendation. I least them here for the author's convenience:

p.1, item (1): this claim is too bold at this stage, before showing the relation between string and LQG variables. Strings do not follow the same dynamical equations of holonomies and do not live on the same space.

p.2, item (4): in string theory matter is not added by hand, it is part of the string spectrum and depends on the compactification scheme.

Several typos should be corrected (ref. [7] incomplete, "that fact" below (18), 1st line p.8 "D+1" should be "D", p.10 "extermization", etc.).

\end{document}